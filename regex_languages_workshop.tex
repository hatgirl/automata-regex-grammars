\documentclass{beamer}
\mode<presentation>{\usetheme{CambridgeUS}}
\usecolortheme{beaver}
\usepackage{caption}
\beamertemplatenavigationsymbolsempty
\usepackage{fancybox}
\usepackage{csquotes}

\newenvironment{myfont}{\fontfamily{LinuxLibertineT-OsF}\selectfont}{\par}

\usepackage{pst-node}
\psset{arrowscale=2,arrows=->}
\def\VPh{\vphantom{\displaystyle\sum_{i=n}^m {i^2}}}
\def\psBox#1#2{\psframebox[fillcolor=#1,fillstyle=solid]{\VPh\displaystyle#2}}

\title{Intro to Automata and Languages}
\institute[Recurse Center]{An hour-long explanation of Dave's ``well actually"}
\date{\today}


\begin{document}

\begin{frame}
	\titlepage
\end{frame}

\section*{Outline}
\begin{frame}
	\frametitle{Outline}
	\tableofcontents

\end{frame}

\section*{Languages}
\begin{frame}
	\frametitle{What is a language?}
	A language is a set of strings over some alphabet
	\begin{itemize}
		\item $\Sigma $ denotes the alphabet
		\item $\mathcal{L}$ denotes the language over that alphabet	
	\end{itemize}
	
	\vspace{0.3in}
	Examples:\\
	\begin{itemize}
	\item $\Sigma = \{0, 1\}$, $\mathcal{L} = \{ 01, 10, 101, 010 \}$\\
	\item $\Sigma = \{0, 1\}$, $\mathcal{L} = \Sigma ^{*}$ \\
	\item $\Sigma = \{a, b, c\}$, $\mathcal{L} = a^* b^* c^* $ \\
	\item $\Sigma = \{Expr, Operator, Open, Close\}$, $\mathcal{L} = \{Expr, Expr\; Operator\; Expr, Open\; Expr\; Close \}$
	\end{itemize}
\end{frame}

\section*{Finite Automata}
\begin{frame}
	\frametitle{What are automata?}
	The formal definition of a finite automaton is a 5-tuple 
	$(Q, \Sigma, \delta, q_0, F)$ where:
	
	\begin{itemize}
	\item $Q = $ no one cares
	\item $\Sigma = $ already talked about this
	\item $\delta: Q \times \Sigma \rightarrow Q  = $ you will never use this 
	\item $q_0 = $  lame
	\item $F = $ whatever
	\end{itemize}
	
\end{frame}

\begin{frame}
	\frametitle{What are deterministic finite automata?}

	\begin{figure}
		\includegraphics[scale=0.3]{introdfa.png}
	\end{figure}	
\end{frame}


\begin{frame}
	\frametitle{What are non-deterministic finite automata?}
	
	\begin{figure}
		\includegraphics[scale=0.6]{nondblacksheep.png}
	\end{figure}

\end{frame}

\begin{frame}
	\frametitle{Do they describe different languages?}
\centering \large{\textbf{NO}}

\end{frame}

%------

\begin{frame}

Time for exercises!

\vspace{0.2in}

1. Draw a DFA that describes the language $\mathcal{L} = \{ x \in \{a, b\}^* | x \text{ contains ab as a substring} \}$

\pause

\begin{figure}
\includegraphics[scale=0.5]{absub.png}
\end{figure}

\end{frame}

%-----------------

\begin{frame}

Time for exercises!

\vspace{0.2in}

2. Draw a DFA that describes the language $\mathcal{L} = \{ x \in \{0,1\}^* | x \text{ has an odd number of 1's and an even number of 0's} \}$

\pause

\begin{figure}
\includegraphics[scale=0.4]{evenodd.png}
\end{figure}

\end{frame}


%----------------------------------------------------------

\section*{Automata $\iff$ Regular Languages}

\begin{frame}
	\frametitle{Automata $\rightarrow $ regex}
	\begin{figure}
	\centering \includegraphics[scale=0.43]{rigour.png}
	\end{figure}
\end{frame}	
	

\begin{frame}
	\frametitle{Automata $\rightarrow $ regex}
	\begin{figure}
	\centering \includegraphics[scale=0.43]{rigor1.png}
	\end{figure}
\end{frame}	
		

\begin{frame}
	\frametitle{Automata $\rightarrow $ regex}
	\begin{figure}
	\centering \includegraphics[scale=0.43]{rigour2.png}
	\end{figure}
\end{frame}	

\begin{frame}
	\frametitle{Automata $\rightarrow $ regex}
	\begin{figure}
	\centering \includegraphics[scale=0.43]{rigour3.png}
	\end{figure}
\end{frame}	

\begin{frame}
	\frametitle{Automata $\rightarrow $ regex}
	\begin{figure}
	\centering \includegraphics[scale=0.43]{rigour4.png}
	\end{figure}
\end{frame}			
	
	
\begin{frame}
	\frametitle{Automata $\rightarrow $ regex}
	\begin{figure}
	\centering \includegraphics[scale=0.43]{rigour5.png}
	\end{figure}
	
	\pause 
	
	[Rr]igo(ur$|$r)
	
	\pause
		
	[Rr]igou?r \\
\end{frame}			


\begin{frame}
	\frametitle{Automata $\rightarrow $ regex}
	\vspace{-0.4in}
	\begin{figure}
	\centering \includegraphics[scale=0.31]{emails.png}
	\end{figure}
	
	\pause
	\vspace{-0.2in}
	Matches email addresses 
	

\string^([a-z 0-9\_\textbackslash. -]$+$)@([\textbackslash da-z\textbackslash. -]$+$)\textbackslash.([a-z\textbackslash.]\{2,6\})\$

\end{frame}

%-------------------------------------------------------------------

\section*{What can't finite automata do?}

\begin{frame}
\frametitle{Things finite automata cannot do}

\begin{itemize}
\item Match balanced parentheses
\vspace{0.2in}
\item Match expressions that are made out of combining expressions that were made out of combining expressions that ....
\vspace{0.2in}
\item Match palindromes
\vspace{0.2in}
\item Match composite numbers
\end{itemize}

\end{frame}


\begin{frame}
	\frametitle{So about that ``well actually"}
	\begin{center}
In much of the computing world \\
\textbf{Regular Expressions $\neq $ Regular Languages}

	\end{center}
	\vspace{0.2in}
	People who might assume equality:\\
	\begin{itemize}
	\item computer science theory professors
	\item computer science theory enthusiasts
	\item computer science theory students
	\item computer science theory [a-z]+
	\end{itemize}
\end{frame}


\begin{frame}
\frametitle{So about that ``well actually"}
	\begin{center}
	``Well actually, Tom, because many regular expression engines support back references, they can represent context-free languages too,"\\
	- Dave Albert
	\end{center}
\end{frame}

\begin{frame}
\frametitle{Regexs = regular languages + other stuff}
\begin{center}
The language of composite numbers is not regular...


\vspace{0.2in}

... but most regex engines can totally match them (in unary).

\vspace{0.2in}

Is this not how everyone spends their Friday nights? 
\end{center}
\end{frame}

\begin{frame}
\frametitle{Regexs = regular languages + other stuff}
\begin{center}
Try it! \\
\vspace{0.2in}
r' \string^1?\$ $|$ \string^(11$+$?)\textbackslash 1$+$\$'
\end{center}

\vspace{0.2in}
\pause
The first part of the `or', \string^1?\$, matches 0 or 1 as not prime.\\
\pause
\vspace{0.2in}
The second part matches divisors
\end{frame}



%-------------------------------------------------------------------

\section{Summary}
\begin{frame}
\frametitle{In conclusion...}
\begin{itemize}
\item Automata: pretty cool
\vspace{0.2in}
\item Regex: also pretty cool
\vspace{0.2in}
\item Regular languages: not regex... except for when they are
\end{itemize}
\end{frame}
\end{document}