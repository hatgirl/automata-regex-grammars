\documentclass{beamer}
\mode<presentation>{\usetheme{CambridgeUS}}
\usecolortheme{beaver}
\usepackage{caption}
\beamertemplatenavigationsymbolsempty
\usepackage{fancybox}
\usepackage{csquotes}

\usepackage{pst-node}
\psset{arrowscale=2,arrows=->}
\def\VPh{\vphantom{\displaystyle\sum_{i=n}^m {i^2}}}
\def\psBox#1#2{\psframebox[fillcolor=#1,fillstyle=solid]{\VPh\displaystyle#2}}

\title{Intro to Automata and Languages}
\institute[]{An hour-long explanation of Dave's ``Well actually"}
\date{\today}


\begin{document}

\begin{frame}
	\titlepage
\end{frame}

\section*{Outline}
\begin{frame}
	\frametitle{Outline}
	\tableofcontents

\end{frame}

\section*{Preliminaries}
\begin{frame}
	\frametitle{What is a language?}
	

\end{frame}

\section*{Finite Automata}
\begin{frame}
	\frametitle{What are automata?}
	The formal definition of a finite automaton is a 5-tuple 
	$(Q, \Sigma, \delta, q_0, F)$ where:
	
	\begin{itemize}
	\item $Q = $ no one cares
	\item $\Sigma = $ this one's actually kind of useful, it denotes the alphabet being used
	\item $\delta: Q \times \Sigma \rightarrow Q  = $ you will never use this 
	\item $q_0 = $  lame
	\item $F = $ whatever
	\end{itemize}
	
\end{frame}

\begin{frame}
	\frametitle{What are automata?}

	\begin{figure}
		\includegraphics[scale=0.3]{introdfa.png}
	\end{figure}	
\end{frame}

%-------------------------------------------------------------------

\section*{Regular Languages}

\begin{frame}
	\frametitle{What are regular languages?}
\end{frame}



%-------------------------------------------------------------------


\section*{Automata $\iff$ Regular Languages}
\begin{frame}
	\frametitle{From automata to regex}
\end{frame}
%-------------------------------------------------------------------


\section*{What can('t) RegExs do?}

\begin{frame}
	\frametitle{So about that ``well actually"...}
\end{frame}

\begin{frame}
	\frametitle{But we still need more}
\end{frame}

%-------------------------------------------------------------------

\end{document}